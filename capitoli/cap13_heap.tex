\chapter{Heap Priority Queue}
\label{cap:heap}

\begin{lstlisting}[
    language=Python,
    caption={Implementazione di HeapPriorityQueue: gestione di un heap binario tramite array con algoritmi di upheap e downheap per il bilanciamento.},
    label={lst:HeapPriorityQueue}
]
from .priority_queue_base import PriorityQueueBase

class Empty(Exception):
  pass


class HeapPriorityQueue(PriorityQueueBase): # base class defines _Item
  """A min-oriented priority queue implemented with a binary heap."""

  #------------------------------ nonpublic behaviors ------------------------------
  def _parent(self, j):
    return (j-1) // 2

  def _left(self, j):
    return 2*j + 1

  def _right(self, j):
    return 2*j + 2

  def _has_left(self, j):
    return self._left(j) < len(self._data)     # index beyond end of list?

  def _has_right(self, j):
    return self._right(j) < len(self._data)    # index beyond end of list?

  def _swap(self, i, j):
    """Swap the elements at indices i and j of array."""
    self._data[i], self._data[j] = self._data[j], self._data[i]
  
  \end{lstlisting}  

  \clearpage
  \begin{lstlisting}[
    language=Python,
    firstnumber=29
  ]

  def _upheap(self, j):
    parent = self._parent(j)
    if j > 0 and self._data[j] < self._data[parent]:
      self._swap(j, parent)
      self._upheap(parent)             # recur at position of parent

  def _downheap(self, j):
    if self._has_left(j):
      left = self._left(j)
      small_child = left               # although right may be smaller
      if self._has_right(j):
        right = self._right(j)
        if self._data[right] < self._data[left]:
          small_child = right
      if self._data[small_child] < self._data[j]:
        self._swap(j, small_child)
        self._downheap(small_child)    # recur at position of small child

  def _heapify(self):
    """Bottom-up construction of a heap in O(n) time."""
    # start at the parent of the last element
    start = self._parent(len(self._data) - 1)
    for j in range(start, -1, -1): # go backward from start to 0
      self._downheap(j)

  #------------------------------ public behaviors ------------------------------
  def __init__(self, contents=()):
    """Create a new empty Priority Queue.
    
    If contents is provided, it should be an iterable of (k,v) tuples.
    """
    self._data = [self._Item(k, v) for k, v in contents]
    if len(self._data) > 1:
      self._heapify()

  """ without Heapify
  def __init__(self):
    Create a new empty Priority Queue.
    self._data = []
  """

  def __len__(self):
    """Return the number of items in the priority queue."""
    return len(self._data)
  
  \end{lstlisting}  
  
  \clearpage
  \begin{lstlisting}[
    language=Python,
    firstnumber=74
  ]

  def add(self, key, value):
    """Add a key-value pair to the priority queue."""
    self._data.append(self._Item(key, value))
    self._upheap(len(self._data) - 1)            # upheap newly added position

  def min(self):
    """Return but do not remove (k,v) tuple with minimum key.

    Raise Empty exception if empty.
    """
    if self.is_empty():
      raise Empty('Priority queue is empty.')
    item = self._data[0]
    return (item._key, item._value)

  def remove_min(self):
    """Remove and return (k,v) tuple with minimum key.

    Raise Empty exception if empty.
    """
    if self.is_empty():
      raise Empty('Priority queue is empty.')
    self._swap(0, len(self._data) - 1)           # put minimum item at the end
    item = self._data.pop()                      # and remove it from the list;
    self._downheap(0)                            # then fix new root
    return (item._key, item._value)

\end{lstlisting}
