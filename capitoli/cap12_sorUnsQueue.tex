\chapter{Sorted Unsorted Priority Queue}
\label{cap:sorUnsQueue}

\begin{lstlisting}[
    language=Python,
    caption={Implementazione di SortedPriorityQueue.},
    label={lst:SortedPriorityQueue}
]
from .priority_queue_base import PriorityQueueBase
from ..list.positional_list import PositionalList

class Empty(Exception):
  pass

class SortedPriorityQueue(PriorityQueueBase): # base class defines _Item
  """A min-oriented priority queue implemented with a sorted list."""

  #------------------------------ public behaviors ------------------------------
  def __init__(self):
    """Create a new empty Priority Queue."""
    self._data = PositionalList()

  def __len__(self):
    """Return the number of items in the priority queue."""
    return len(self._data)

  def add(self, key, value):
    """Add a key-value pair."""
    newest = self._Item(key, value)             # make new item instance
    walk = self._data.last()       # walk backward looking for smaller key
    while walk is not None and newest < walk.element():
      walk = self._data.before(walk)
    if walk is None:
      self._data.add_first(newest)              # new key is smallest
    else:
      self._data.add_after(walk, newest)        # newest goes after walk
  
  \end{lstlisting}  

  \clearpage
  \begin{lstlisting}[
    language=Python,
    firstnumber=29
  ]

  def min(self):
    """Return but do not remove (k,v) tuple with minimum key.

    Raise Empty exception if empty.
    """
    if self.is_empty():
      raise Empty('Priority queue is empty.')
    p = self._data.first()
    item = p.element()
    return (item._key, item._value)

  def remove_min(self):
    """Remove and return (k,v) tuple with minimum key.

    Raise Empty exception if empty.
    """
    if self.is_empty():
      raise Empty('Priority queue is empty.')
    item = self._data.delete(self._data.first())
    return (item._key, item._value)

\end{lstlisting}


\begin{lstlisting}[
    language=Python,
    caption={Implementazione di UnsortedPriorityQueue.},
    label={lst:UnsortedPriorityQueue}
]
from .priority_queue_base import PriorityQueueBase
from ..list.positional_list import PositionalList

class Empty(Exception):
  pass


class UnsortedPriorityQueue(PriorityQueueBase): # base class defines _Item
  """A min-oriented priority queue implemented with an unsorted list."""

  #----------------------------- nonpublic behavior -----------------------------
  def _find_min(self):
    """Return Position of item with minimum key."""
    if self.is_empty():               # is_empty inherited from base class
      raise Empty('Priority queue is empty')
    small = self._data.first()
    walk = self._data.after(small)
    while walk is not None:
      if walk.element() < small.element():
        small = walk
      walk = self._data.after(walk)
    return small

  \end{lstlisting}  

  \clearpage
  \begin{lstlisting}[
    language=Python,
    firstnumber=23
  ]

  #------------------------------ public behaviors ------------------------------
  def __init__(self):
    """Create a new empty Priority Queue."""
    self._data = PositionalList()

  def __len__(self):
    """Return the number of items in the priority queue."""
    return len(self._data)

  def add(self, key, value):
    """Add a key-value pair."""
    self._data.add_last(self._Item(key, value))

  def min(self):
    """Return but do not remove (k,v) tuple with minimum key.

    Raise Empty exception if empty.
    """
    p = self._find_min()
    item = p.element()
    return (item._key, item._value)

  def remove_min(self):
    """Remove and return (k,v) tuple with minimum key.

    Raise Empty exception if empty.
    """
    p = self._find_min()
    item = self._data.delete(p)
    return (item._key, item._value)

\end{lstlisting}
