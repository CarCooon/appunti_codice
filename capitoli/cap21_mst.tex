\chapter{Minimum Spanning Tree}
\label{cap:mst}

\begin{lstlisting}[
    language=Python,
    caption={Struttura dati Partition (Union-Find): gestione di insiemi disgiunti con compressione del cammino e unione per dimensione.},
    label={lst:Partition}
]

class Partition:
  """Union-find structure for maintaining disjoint sets."""

  #------------------------- nested Position class -------------------------
  class Position:
    __slots__ = '_container', '_element', '_size', '_parent'

    def __init__(self, container, e):
      """Create a new position that is the leader of its own group."""
      self._container = container         # reference to Partition instance
      self._element = e
      self._size = 1
      self._parent = self                 # convention for a group leader

    def element(self):
      """Return element stored at this position."""
      return self._element

  #------------------------- nonpublic utility -------------------------
  def _validate(self, p):
    if not isinstance(p, self.Position):
      raise TypeError('p must be proper Position type')
    if p._container is not self:
      raise ValueError('p does not belong to this container')

  \end{lstlisting}  

  \clearpage
  \begin{lstlisting}[
    language=Python,
    firstnumber=26
  ]

  #------------------------- public Partition methods -------------------------
  def make_group(self, e):
    """Makes a new group containing element e, and returns its Position."""
    return self.Position(self, e)

  def find(self, p):
    """Finds the group containging p and return the position of its leader."""
    self._validate(p)
    if p._parent != p:
      p._parent = self.find(p._parent)    # overwrite p._parent after recursion
    return p._parent

  def union(self, p, q):
    """Merges the groups containg elements p and q (if distinct)."""
    a = self.find(p)
    b = self.find(q)
    if a is not b:                        # only merge if different groups
      if a._size > b._size:
        b._parent = a
        a._size += b._size
      else:
        a._parent = b
        b._size += a._size

  \end{lstlisting}  

\clearpage
\begin{lstlisting}[
  language=Python,
  caption={Algoritmo di Prim: espansione del cluster tramite coda di priorità adattabile.},
  label={lst:MST_Prim},
  firstnumber=0
]
        

from ..priority_queue.heap_priority_queue import HeapPriorityQueue
from ..priority_queue.adaptable_heap_priority_queue import AdaptableHeapPriorityQueue
from .partition import Partition

def MST_Prim(g):
  """Compute a minimum spanning tree of weighted graph g.

  Return a list of edges that comprise the MST (in arbitrary order).
  """
  d = {}                               # d[v] is bound on distance to tree
  tree = []                            # list of edges in spanning tree
  pq = AdaptableHeapPriorityQueue()   # d[v] maps to value (v, e=(u,v))
  pqlocator = {}                       # map from vertex to its pq locator

  # for each vertex v of the graph, add an entry to the priority queue, with
  # the source having distance 0 and all others having infinite distance
  for v in g.vertices():
    if len(d) == 0:                                 # this is the first node
      d[v] = 0                                      # make it the root
    else:
      d[v] = float('inf')                           # positive infinity
    pqlocator[v] = pq.add(d[v], (v,None))

  while not pq.is_empty():
    key,value = pq.remove_min()
    u,edge = value                                  # unpack tuple from pq
    del pqlocator[u]                                # u is no longer in pq
    if edge is not None:
      tree.append(edge)                             # add edge to tree
    for link in g.incident_edges(u):
      v = link.opposite(u)
      if v in pqlocator:                            # thus v not yet in tree
        # see if edge (u,v) better connects v to the growing tree
        wgt = link.element()
        if wgt < d[v]:                              # better edge to v?
          d[v] = wgt                                # update the distance
          pq.update(pqlocator[v], d[v], (v, link))  # update the pq entry

  return tree

  \end{lstlisting}  

\clearpage
\begin{lstlisting}[
  language=Python,
  caption={Algoritmo di Kruskal: approccio greedy basato sull'ordinamento degli archi e gestione dei cicli tramite Partition.},
  label={lst:MST_Kruskal},
  firstnumber=41
]

def MST_Kruskal(g):
  """Compute a minimum spanning tree of a graph using Kruskal's algorithm.

  Return a list of edges that comprise the MST.

  The elements of the graph's edges are assumed to be weights.
  """
  tree = []                   # list of edges in spanning tree
  pq = HeapPriorityQueue()    # entries are edges in G, with weights as key
  forest = Partition()        # keeps track of forest clusters
  position = {}               # map each node to its Partition entry

  for v in g.vertices():
    position[v] = forest.make_group(v)

  for e in g.edges():
    pq.add(e.element(), e)    # edge's element is assumed to be its weight

  size = g.vertex_count()
  while len(tree) != size - 1 and not pq.is_empty():
    # tree not spanning and unprocessed edges remain
    weight,edge = pq.remove_min()
    u,v = edge.endpoints()
    a = forest.find(position[u])
    b = forest.find(position[v])
    if a != b:
      tree.append(edge)
      forest.union(a,b)

  return tree

\end{lstlisting}
